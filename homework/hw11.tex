\section*{Homework 11}

\paragraph{Homework 11.}
In the example given after Theorem 8 as a simple application of Bayes’ law, we did not mathematically
define our probability spaces. Explicitly give the probability spaces we implicitly assumed in which
$\Pr(E_3)$, $\Pr(E_2 \mid A)$, and $\Pr(A \mid E_1)$ are valid probabilities, that is, the original probability
space and the ones induced by the conditional probabilities given $A$ as well as those given $E_1$.

\medskip

\begin{proof}[解答]
    定理 8 の後の例では,次の状況を考えていた.コインが 3 枚あり,そのうち 2 枚は
    公平で,残り 1 枚は表が出る確率 $4/5$ の偏ったコインである.どのコインが偏っているかは
    事前には分からず,どれが偏っているかは一様に確からしいと仮定する.
    $i=1,2,3$ について,$E_i$ を「$i$ 番目のコインが偏っている」という事象とすると,
    \[
        \Pr(E_1)=\Pr(E_2)=\Pr(E_3)=\frac13
    \]
    である.また,3 枚のコインを 1 回ずつ独立に投げ,その結果が
    「1 枚目が裏,2 枚目が裏,3 枚目が表」である事象を $A$ と書いていた.

    この状況を形式的な確率空間として書き下す.

    \medskip
    \noindent
    \textbf{(1) 元の確率空間 $(\Omega,\mathcal{F},\Pr)$}

    コインの「どれが偏っているか」と「各コインの表裏の結果」をまとめて 1 点とする.
    状態を
    \[
        i \in \{1,2,3\}
    \]
    で「$i$ 番目のコインが偏っている」と表し,各コインの結果を
    \[
        x_1,x_2,x_3 \in \{H,T\}
    \]
    ($H$ は表,$T$ は裏)と書く.このとき標本空間を
    \[
        \Omega = \{(i,x_1,x_2,x_3) \mid i\in\{1,2,3\},\ x_1,x_2,x_3\in\{H,T\}\}
    \]
    と定める.$\sigma$-加法族は有限集合なので
    \[
        \mathcal{F} = 2^\Omega
    \]
    とすればよい.

    確率測度 $\Pr$ は,事前分布 $\Pr(E_i)=1/3$ と条件付き確率
    \[
        \Pr(\text{表} \mid \text{偏ったコイン}) = \frac45,\quad
        \Pr(\text{裏} \mid \text{偏ったコイン}) = \frac15,
    \]
    \[
        \Pr(\text{表} \mid \text{公平なコイン}) = \Pr(\text{裏} \mid \text{公平なコイン}) = \frac12
    \]
    から
    \[
        \Pr(\{(i,x_1,x_2,x_3)\})
        = \Pr(E_i)\prod_{j=1}^3 p_{i,j}(x_j)
    \]
    で定める.ここで $p_{i,j}(H)$ は
    $j=i$ なら $4/5$,$j\neq i$ なら $1/2$,$p_{i,j}(T)$ は
    $j=i$ なら $1/5$,$j\neq i$ なら $1/2$ とする.

    このとき,例で使われていた事象は
    \[
        E_k = \{(i,x_1,x_2,x_3)\in\Omega \mid i=k\}
        \quad (k=1,2,3),
    \]
    \[
        A = \{(i,x_1,x_2,x_3)\in\Omega \mid x_1=T,\ x_2=T,\ x_3=H\}
    \]
    と書ける.特に $\Pr(E_3)=1/3$ はこの元の確率空間 $(\Omega,\mathcal{F},\Pr)$ における
    通常の確率である.

    \medskip
    \noindent
    \textbf{(2) 事象 $A$ による条件付き確率から誘導される確率空間}

    つぎに,「結果が $A$ であった」という条件のもとでの確率を考える.Homework 9 と同様に,
    \[
        \Pr_A(F) := \Pr(F \mid A) = \frac{\Pr(F\cap A)}{\Pr(A)} \qquad(F\in\mathcal{F})
    \]
    と定めると,$(\Omega,\mathcal{F},\Pr_A)$ は確率空間になる.この確率空間において
    \[
        \Pr(E_2 \mid A) = \Pr_A(E_2)
    \]
    と解釈していることになる.(必要なら $(A,\mathcal{F}_A,\Pr_A)$ という
    制限された空間を考えてもよいが,ここでは $\Omega$ のままでもよい.)

    \medskip
    \noindent
    \textbf{(3) 事象 $E_1$ による条件付き確率から誘導される確率空間}

    同様に,「1 枚目のコインが偏っている」(すなわち $E_1$ が起きている)という条件のもとでの
    確率を
    \[
        \Pr_{E_1}(F) := \Pr(F \mid E_1) = \frac{\Pr(F\cap E_1)}{\Pr(E_1)} \qquad(F\in\mathcal{F})
    \]
    と定める.すると $(\Omega,\mathcal{F},\Pr_{E_1})$ も確率空間になり,
    この空間において
    \[
        \Pr(A \mid E_1) = \Pr_{E_1}(A)
    \]
    として理解している.(ここでも $(E_1,\mathcal{F}_{E_1},\Pr_{E_1})$ に制限した空間を
    とってもよい.)

    以上により,例の中で暗黙に使っていた
    \[
        \Pr(E_3),\quad \Pr(E_2\mid A),\quad \Pr(A\mid E_1)
    \]
    がそれぞれ意味を持つ 3 つの確率空間,
    すなわち元の $(\Omega,\mathcal{F},\Pr)$ と,
    事象 $A$ による条件付き確率から誘導される $(\Omega,\mathcal{F},\Pr_A)$,
    事象 $E_1$ による条件付き確率から誘導される $(\Omega,\mathcal{F},\Pr_{E_1})$
    を明示的に与えることができた.
\end{proof}
