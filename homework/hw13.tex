\section*{Homework 13}

\paragraph{Homework 13.}
Let $X$ be a $(\Psi,\mathcal{E})$-valued random variable on a probability space
$(\Omega,\mathcal{F},\Pr)$. Define $\Pr_X\colon \mathcal{E} \to [0,1]$
by $\Pr_X(E) = \Pr(X^{-1}(E))$ for any $E\in\mathcal{E}$.
Prove that $(\Psi,\mathcal{E},\Pr_X)$ is a probability space.

\medskip

\begin{proof}[解答]
    $(\Omega,\mathcal{F},\Pr)$ を確率空間とし,
    $X\colon \Omega \to \Psi$ を写像とする。
    仮定より,任意の $E \in \mathcal{E}$ について逆像
    \[
        X^{-1}(E) = \{\omega \in \Omega \mid X(\omega) \in E\}
    \]
    が $\mathcal{F}$ に属しているとする。
    このとき
    \[
        \Pr_X(E) := \Pr\bigl(X^{-1}(E)\bigr)
        \qquad (E \in \mathcal{E})
    \]
    と定めると,$(\Psi,\mathcal{E},\Pr_X)$ が確率空間になることを示す。
    つまり,$\Pr_X$ が $\mathcal{E}$ 上の確率測度になっていることを確かめればよい。

    \medskip
    \noindent
    (1) \textbf{非負性}

    任意の $E\in\mathcal{E}$ に対し,$X^{-1}(E)\in\mathcal{F}$ かつ
    $\Pr$ は確率測度なので
    \[
        \Pr_X(E) = \Pr\bigl(X^{-1}(E)\bigr) \ge 0
    \]
    が成り立つ。

    \medskip
    \noindent
    (2) \textbf{全体集合の確率が 1 であること}

    $\Psi$ の逆像は
    \[
        X^{-1}(\Psi) = \Omega
    \]
    であるから,
    \[
        \Pr_X(\Psi)
        = \Pr\bigl(X^{-1}(\Psi)\bigr)
        = \Pr(\Omega)
        = 1
    \]
    となる。

    \medskip
    \noindent
    (3) \textbf{可算加法性}

    $(E_i)_{i\ge1}$ を $\mathcal{E}$ の互いに素な($i\ne j$ なら $E_i\cap E_j=\varnothing$)
    列とする。このとき逆像の基本的性質より
    \[
        X^{-1}\Bigl(\bigcup_{i=1}^\infty E_i\Bigr)
        =
        \bigcup_{i=1}^\infty X^{-1}(E_i)
    \]
    が成り立つ。また,もし $i\ne j$ で
    \[
        X^{-1}(E_i) \cap X^{-1}(E_j) \ne \varnothing
    \]
    だとすると,ある $\omega$ が存在して
    $\omega \in X^{-1}(E_i) \cap X^{-1}(E_j)$ となるが,
    これは $X(\omega)\in E_i$ かつ $X(\omega)\in E_j$ を意味し,
    $E_i$ と $E_j$ が互いに素であることに反する。
    したがって $X^{-1}(E_i)$ 同士も互いに素である。

    よって,$\Pr$ の可算加法性より
    \begin{align*}
        \Pr_X\Bigl(\bigcup_{i=1}^\infty E_i\Bigr)
         & = \Pr\Bigl(
        X^{-1}\Bigl(\bigcup_{i=1}^\infty E_i\Bigr)
        \Bigr)                                            \\
         & = \Pr\Bigl(
        \bigcup_{i=1}^\infty X^{-1}(E_i)
        \Bigr)                                            \\
         & = \sum_{i=1}^\infty \Pr\bigl(X^{-1}(E_i)\bigr) \\
         & = \sum_{i=1}^\infty \Pr_X(E_i).
    \end{align*}

    \medskip

    (1)〜(3) から,$\Pr_X$ は $\mathcal{E}$ 上の確率測度である。
    したがって $(\Psi,\mathcal{E},\Pr_X)$ は確率空間になっている。
\end{proof}
