\section*{Homework 14}

\paragraph{Homework 14.}
Give a random variable $X$ whose expectation is not well-defined, that is, $E(X)$ does not exist. (Hint:
This does not mean that the expectation is infinite. Recall that for $E(X)$ to exist, $E(X^+)$ and/or $E(X^-)$ should con-
verge.)

\medskip

\begin{proof}[解答]
    整数を使った簡単な例を考える。

    まず標本空間と確率のつけ方を次のように決める:
    \[
        \Omega = \mathbb{Z}\setminus\{0\},\qquad
        \mathcal{F} = 2^\Omega.
    \]
    各 $n\in\Omega$ に対して
    \[
        \Pr(\{n\}) = \frac{C}{n^2}
    \]
    とおく。ただし $C>0$ は
    \[
        \sum_{n\in\Omega} \Pr(\{n\}) = 1
    \]
    となるようにとった定数である($\sum_{n=1}^\infty 1/n^2$ が有限であることから,
    そのような $C$ が取れることは知られている)。このとき
    $(\Omega,\mathcal{F},\Pr)$ は確率空間になっている。

    この確率空間の上で写像 $X\colon\Omega\to\mathbb{R}$ を
    \[
        X(n) = n \qquad (n\in\Omega)
    \]
    と定める。つまり,「$n$ が出たら値として $n$ をとる」確率変数である。

    \medskip
    $X$ の期待値を形式的に
    \[
        E(X)
        = \sum_{n\in\Omega} X(n)\Pr(\{n\})
        = \sum_{n\in\Omega} n\,\Pr(\{n\})
    \]
    と書くと,正の整数と負の整数に分けて
    \[
        E(X)
        = \sum_{n=1}^\infty n\,\Pr(\{n\})
        + \sum_{n=-\infty}^{-1} n\,\Pr(\{n\})
    \]
    となる。

    まず正の部分を計算する:
    \[
        \sum_{n=1}^\infty n\,\Pr(\{n\})
        = \sum_{n=1}^\infty n \cdot \frac{C}{n^2}
        = C \sum_{n=1}^\infty \frac{1}{n}.
    \]
    ここで $\sum_{n=1}^\infty 1/n$ は調和級数であり,無限大に発散するので,
    この和は $+\infty$ になる。

    次に負の部分を計算する。$n=-m$ と書き直すと
    \begin{align*}
        \sum_{n=-\infty}^{-1} n\,\Pr(\{n\})
         & = \sum_{m=1}^\infty (-m)\,\Pr(\{-m\})       \\
         & = \sum_{m=1}^\infty (-m)\cdot \frac{C}{m^2} \\
         & = -C \sum_{m=1}^\infty \frac{1}{m}
        = -\infty
    \end{align*}
    となる。

    したがって,$X$ の「正の部分に対応する和」は $+\infty$,
    「負の部分に対応する和」は $-\infty$ になってしまう。
    形式的には
    \[
        E(X)
        = \bigl(+\infty\bigr) + \bigl(-\infty\bigr)
    \]
    という形になり,値を 1 つに決めることができない。

    期待値が「存在する」と言うためには,
    正の側と負の側の和がそれぞれきちんと収束している必要がある。
    この例では,一方は $+\infty$,もう一方は $-\infty$ に発散しているので,
    $E(X)$ は定まらず,「期待値が存在しない」確率変数の具体例になっている。
\end{proof}
