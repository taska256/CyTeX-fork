\section*{Homework 8}

\paragraph{Homework 8.}
Give an example of pairwise independent events which are not mutually independent.

\medskip

\begin{proof}[解答]
    次のような確率空間を考える.標本空間を
    \[
        \Omega = \{1,2,3,4\}
    \]
    とし,一様分布
    \[
        \Pr(\{1\}) = \Pr(\{2\}) = \Pr(\{3\}) = \Pr(\{4\}) = \frac14
    \]
    を入れる.このとき
    \[
        A = \{1,2\},\qquad
        B = \{1,3\},\qquad
        C = \{1,4\}
    \]
    とおく.

    まず各事象の確率は
    \[
        \Pr(A) = \Pr(B) = \Pr(C) = \frac12
    \]
    である.また 2 つずつの共通部分は
    \[
        A \cap B = \{1\},\quad
        A \cap C = \{1\},\quad
        B \cap C = \{1\}
    \]
    なので
    \[
        \Pr(A \cap B)
        = \Pr(A \cap C)
        = \Pr(B \cap C)
        = \frac14
        = \frac12 \cdot \frac12.
    \]
    したがって,$A,B,C$ はどの 2 つをとっても
    \[
        \Pr(\text{積}) = \Pr(\text{片方}) \Pr(\text{もう片方})
    \]
    が成り立ち,pairwise independent である.

    一方,3つすべての共通部分は
    \[
        A \cap B \cap C = \{1\}
    \]
    であるから
    \[
        \Pr(A \cap B \cap C) = \frac14
    \]
    である.しかし
    \[
        \Pr(A)\Pr(B)\Pr(C)
        = \frac12 \cdot \frac12 \cdot \frac12
        = \frac18
    \]
    となり,
    \[
        \Pr(A \cap B \cap C) \ne \Pr(A)\Pr(B)\Pr(C)
    \]
    が成り立つ.したがって $A,B,C$ は互いに独立(mutually independent)
    ではない.

    以上により,$A,B,C$ は pairwise independent だが mutually independent ではない事象の具体例になっている.
\end{proof}
