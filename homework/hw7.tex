\section*{Homework 7}

\paragraph{Homework 7.}
Prove Theorem 4. (Hint: Notice that any event is a disjoint union of simple events.
Consider the contribution of each simple event on each side of the equation and use
the fact $0 = (1-1)^t = \sum_{i=0}^t (-1)^i \binom{t}{i}$, which is an easy corollary
of the binomial theorem.)

\medskip

\begin{proof}[解答]
    Theorem 4(包除原理)は,確率空間 $(\Omega,\mathcal{F},\Pr)$ と
    イベント $E_0,E_1,\dots,E_{n-1}\in\mathcal{F}$ に対して
    \[
        \Pr\Bigl(\bigcup_{i=0}^{n-1} E_i\Bigr)
        =
        \sum_{i} \Pr(E_i)
        - \sum_{i<j} \Pr(E_i\cap E_j)
        + \sum_{i<j<k} \Pr(E_i\cap E_j\cap E_k)
        - \cdots
    \]
    \[
        \qquad\qquad\qquad
        + (-1)^{\ell+1}
        \sum_{i_1<\cdots<i_\ell}
        \Pr\bigl(E_{i_1}\cap\cdots\cap E_{i_\ell}\bigr)
        + \cdots
    \]
    が成り立つ,という主張である.以下ではヒントに従って証明する.

    まず,$\Omega$ の元 $\omega\in\Omega$ を 1 点だけ含む集合 $\{\omega\}$ を
    「単純事象(simple event)」と呼ぶ.
    任意のイベント $E\in\mathcal{F}$ は
    \[
        E \;=\; \bigcup_{\omega\in E} \{\omega\}
    \]
    と表せ,右辺の和集合は互いに素(disjoint)である.
    したがって,確率測度の加法性より
    \[
        \Pr(E)
        = \sum_{\omega\in E} \Pr(\{\omega\})
    \]
    が成り立つ.ゆえに,示したい等式は「$\Omega$ の各元 $\omega$ について,
    左辺と右辺における $\Pr(\{\omega\})$ の係数が一致する」ことを示せば十分である.

    そこで,ある $\omega\in\Omega$ を固定する.
    この $\omega$ を含むイベントの個数を
    \[
        t = \bigl|\{\,i \mid 0\le i\le n-1,\ \omega\in E_i\,\}\bigr|
    \]
    とおく.($t=0$ の場合,$\omega$ はどの $E_i$ にも属さない場合である.)

    もし $t=0$ ならば,$\omega$ はどの $E_i$ にも属さないので
    $\omega\notin \bigcup_i E_i$ である.したがって
    左辺の $\Pr(\{\omega\})$ の係数は $0$ となる.

    つぎに $t\ge 1$ の場合を考える.
    このとき $\omega$ は少なくとも 1 つの $E_i$ に属するので
    $\omega\in\bigcup_i E_i$ であり,左辺の係数は $1$ である.

    右辺については,$\omega$ を含む $E_i$ が $t$ 個あるので,
    それらから $k$ 個選んだ交わりには必ず $\omega$ が含まれる.
    そのような交わりの個数は $\binom{t}{k}$ 個である.ゆえに右辺での
    $\Pr(\{\omega\})$ の係数は
    \[
        \sum_{k=1}^t (-1)^{k+1} \binom{t}{k}
    \]
    となる.

    二項定理の結果
    \[
        0 = (1-1)^t = \sum_{k=0}^t (-1)^k \binom{t}{k}
    \]
    を用いると
    \[
        \sum_{k=1}^t (-1)^{k+1} \binom{t}{k}
        = 1
    \]
    となる.これは左辺の係数と一致する.

    以上より Theorem 4 が従う。
\end{proof}
