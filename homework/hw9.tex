\section*{Homework 9}

\paragraph{Homework 9.}
Let $(\Omega,\mathcal{F},\Pr)$ be a probability space and $E\in\mathcal{F}$ an event such that
$\Pr(E) > 0$. For any $F\in\mathcal{F}$, define $\Pr_E$ to be the conditional probability
\[
    \Pr_E(F) = \Pr(F \mid E)
\]
of $F$ given $E$. Define $\mathcal{F}_E = \{F \in\mathcal{F} \mid F \subseteq E\}$.
Show that $(\Omega,\mathcal{F},\Pr_E)$ and $(E,\mathcal{F}_E,\Pr_E)$ are both probability
spaces if the domain of $\Pr_E$ is chosen appropriately in each case.

\medskip

\begin{proof}[解答]
    まず,確率空間 $(\Omega,\mathcal{F},\Pr)$ と $E\in\mathcal{F}$ で $\Pr(E)>0$ を固定する.
    このとき任意の $F\in\mathcal{F}$ に対し条件付き確率
    \[
        \Pr_E(F) = \Pr(F \mid E) := \frac{\Pr(F \cap E)}{\Pr(E)}
    \]
    を定義する.

    \medskip
    \noindent
    \textbf{(1) $(\Omega,\mathcal{F},\Pr_E)$ が確率空間になること.}

    ここでは,$\Pr_E$ の定義域を元の $\Pr$ と同じく $\mathcal{F}$ とする.
    すなわち
    \[
        \Pr_E : \mathcal{F} \to [0,1],\qquad
        F \mapsto \frac{\Pr(F \cap E)}{\Pr(E)}
    \]
    とみなす.このとき,以下の 3 条件を確かめればよい.

    \begin{itemize}
        \item[(i)] 非負性:任意の $F\in\mathcal{F}$ について
              \[
                  \Pr_E(F) = \frac{\Pr(F \cap E)}{\Pr(E)} \ge 0
              \]
              が成り立つ.分子・分母ともに非負($\Pr$ は確率測度,かつ $\Pr(E)>0$)であるからである.

        \item[(ii)] 全体集合の確率:$\Omega \in\mathcal{F}$ に対して
              \[
                  \Pr_E(\Omega)
                  = \Pr(\Omega \mid E)
                  = \frac{\Pr(\Omega \cap E)}{\Pr(E)}
                  = \frac{\Pr(E)}{\Pr(E)}
                  = 1
              \]
              が成り立つ.

        \item[(iii)] 可算加法性:$(F_i)_{i\ge 1}$ を $\mathcal{F}$ 上の互いに素な($i\ne j$ なら $F_i\cap F_j=\varnothing$)列とする.
              このとき,$F_i\cap E$ も互いに素であり
              \[
                  \bigcup_{i=1}^\infty (F_i \cap E)
                  = \Bigl(\bigcup_{i=1}^\infty F_i\Bigr) \cap E
              \]
              が成り立つ.したがって元の測度 $\Pr$ の可算加法性より
              \begin{align*}
                  \Pr_E\Bigl(\bigcup_{i=1}^\infty F_i\Bigr)
                   & = \frac{\Pr\bigl((\bigcup_i F_i)\cap E\bigr)}{\Pr(E)} \\
                   & = \frac{\Pr\bigl(\bigcup_i(F_i\cap E)\bigr)}{\Pr(E)}  \\
                   & = \frac{\sum_i \Pr(F_i\cap E)}{\Pr(E)}                \\
                   & = \sum_i \frac{\Pr(F_i\cap E)}{\Pr(E)}                \\
                   & = \sum_i \Pr_E(F_i)
              \end{align*}

              が従う.
    \end{itemize}

    以上より,$(\Omega,\mathcal{F},\Pr_E)$ は確率空間である.

    \medskip
    \noindent
    \textbf{(2) $(E,\mathcal{F}_E,\Pr_E)$ が確率空間になること.}

    つぎに
    \[
        \mathcal{F}_E = \{F\in\mathcal{F} \mid F \subseteq E\}
    \]
    を考える.これは $E$ を全体集合とみなしたときの $\sigma$-加法族である.
    ここでは $\Pr_E$ の定義域を $\mathcal{F}_E$ に制限した
    \[
        \Pr_E : \mathcal{F}_E \to [0,1],\qquad
        F \mapsto \frac{\Pr(F)}{\Pr(E)}
    \]
    を確率測度として考える($F\subseteq E$ なので $F\cap E = F$ であることを用いた).

    このときも,同様に 3 つの性質を確かめる:

    \begin{itemize}
        \item[(i)] 非負性:任意の $F\in\mathcal{F}_E$ について $\Pr(F)\ge0$ かつ $\Pr(E)>0$ より
              \[
                  \Pr_E(F) = \frac{\Pr(F)}{\Pr(E)} \ge 0.
              \]

        \item[(ii)] 全体集合の確率:ここでの全体集合は $E$ であり,$E\in\mathcal{F}_E$ である.よって
              \[
                  \Pr_E(E)
                  = \frac{\Pr(E)}{\Pr(E)}
                  = 1
              \]
              が成り立つ.

        \item[(iii)] 可算加法性:$(F_i)_{i\ge1}$ を $\mathcal{F}_E$ の互いに素な列とする.
              すると $\bigcup_i F_i$ も $E$ の部分集合であり $\mathcal{F}_E$ に属する.
              元の測度 $\Pr$ の可算加法性より
              \[
                  \Pr_E\Bigl(\bigcup_{i=1}^\infty F_i\Bigr)
                  = \frac{\Pr(\bigcup_i F_i)}{\Pr(E)}
                  = \frac{\sum_i \Pr(F_i)}{\Pr(E)}
                  = \sum_i \frac{\Pr(F_i)}{\Pr(E)}
                  = \sum_i \Pr_E(F_i)
              \]
              が従う.
    \end{itemize}

    以上から,$(E,\mathcal{F}_E,\Pr_E)$ も確率空間になっている.

    したがって,$\Pr_E$ の定義域をそれぞれ
    \[
        \text{(1) } \mathcal{F},\qquad
        \text{(2) } \mathcal{F}_E
    \]
    と選べば,$(\Omega,\mathcal{F},\Pr_E)$ および $(E,\mathcal{F}_E,\Pr_E)$ はどちらも確率空間となることが示された.
\end{proof}
