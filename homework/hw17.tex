\section*{Homework 17}

\paragraph{Homework 17.}
Prove Theorem 20.

\medskip

\noindent\textbf{Theorem 20 (Linearity of conditional expectation).}
For any finite set of discrete random variables $X_0,X_1,\dots,X_{n-1}$ with finite expectations on a
discrete probability space $(\Omega,\mathcal{F},\Pr)$ and for any event $E\in\mathcal{F}$ with $\Pr(E)>0$,
it holds that
\[
    E\Bigl(\sum_{i=0}^{n-1} c_i X_i \,\Bigm|\, E\Bigr)
    = \sum_{i=0}^{n-1} c_i\,E(X_i \mid E),
\]
where $c_0,\dots,c_{n-1}\in\mathbb{R}$ are arbitrary constants.

\medskip

\begin{proof}[解答]
    $(\Omega,\mathcal{F},\Pr)$ を離散確率空間とし,$E\in\mathcal{F}$ は $\Pr(E)>0$ を満たす事象とする。
    $X_0,X_1,\dots,X_{n-1}$ は有限期待値をもつ離散確率変数とし,
    $c_0,\dots,c_{n-1}\in\mathbb{R}$ を定数とする。

    まず,新しい確率変数
    \[
        Y = \sum_{i=0}^{n-1} c_i X_i
    \]
    を定める。各 $X_i$ の期待値が有限であり,$c_i$ も有限な定数なので,
    $Y$ の期待値 $E(Y)$ も有限である(有限個の和なので問題はない)。

    Homework 16 で示した離散の場合の条件付き期待値の表示
    \[
        E(Y \mid E)
        = \sum_{\omega\in\Omega} Y(\omega)\,\Pr(\{\omega\}\mid E)
    \]
    を用いると,
    \begin{align*}
        E\Bigl(\sum_{i=0}^{n-1} c_i X_i \,\Bigm|\, E\Bigr)
         & = E(Y \mid E)                                             \\
         & = \sum_{\omega\in\Omega} Y(\omega)\,\Pr(\{\omega\}\mid E) \\
         & = \sum_{\omega\in\Omega}
        \Bigl(\sum_{i=0}^{n-1} c_i X_i(\omega)\Bigr)\Pr(\{\omega\}\mid E).
    \end{align*}

    ここで $\omega$ に関する和の中身を分配法則で展開すると,
    \begin{align*}
        E\Bigl(\sum_{i=0}^{n-1} c_i X_i \,\Bigm|\, E\Bigr)
         & = \sum_{\omega\in\Omega}
        \sum_{i=0}^{n-1} c_i X_i(\omega)\,\Pr(\{\omega\}\mid E).
    \end{align*}
    添字 $i$ は有限個($0,\dots,n-1$)しかないので,
    和の順序を入れ替えることができる:
    \begin{align*}
        E\Bigl(\sum_{i=0}^{n-1} c_i X_i \,\Bigm|\, E\Bigr)
         & = \sum_{i=0}^{n-1} c_i
        \sum_{\omega\in\Omega} X_i(\omega)\,\Pr(\{\omega\}\mid E).
    \end{align*}
    ここで $c_i$ は $\omega$ に依存しない定数なので,内側の和の外に出している。

    さらに,内側の和
    \[
        \sum_{\omega\in\Omega} X_i(\omega)\,\Pr(\{\omega\}\mid E)
    \]
    は,まさに条件付き期待値 $E(X_i\mid E)$ の定義そのものである。したがって
    \[
        E\Bigl(\sum_{i=0}^{n-1} c_i X_i \,\Bigm|\, E\Bigr)
        = \sum_{i=0}^{n-1} c_i\,E(X_i\mid E)
    \]
    が従う。

    これで Theorem~20 の主張が示された。
\end{proof}
