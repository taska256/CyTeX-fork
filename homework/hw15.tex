\section*{Homework 15}

\paragraph{Homework 15.}
Prove Theorem 15.

\medskip

\noindent\textbf{Theorem 15.}
Let $\mathcal{E} \subseteq \mathcal{F}$ be a set of events in a discrete probability space
$(\Omega,\mathcal{F},\Pr)$. For $E \in \mathcal{E}$, define $X_E$ to be the indicator random variable for $E$,
and define
\[
    X = \sum_{E \in \mathcal{E}} X_E.
\]
If
\[
    E(X) < 1,
\]
then
\[
    \Pr(X = 0) > 0,
\]
that is, there is a nonzero probability that none of the events in $\mathcal{E}$ occurs.

\medskip

\begin{proof}[解答]
    各 $E \in \mathcal{E}$ について,指示関数 $X_E$ は
    \[
        X_E(\omega) =
        \begin{cases}
            1 & (\omega \in E\ \text{のとき}),   \\
            0 & (\omega \notin E\ \text{のとき})
        \end{cases}
    \]
    と定められている。したがって
    \[
        X(\omega) = \sum_{E \in \mathcal{E}} X_E(\omega)
    \]
    は,「$\mathcal{E}$ の中でその標本点 $\omega$ のもとで起こっている事象の個数」
    を表している。特に,$X(\omega)$ はいつも $0,1,2,\dots$ のいずれかの値をとる
    ($0$ 以上の整数値をとる)確率変数である。

    この性質を用いて期待値 $E(X)$ を調べる。
    $X$ がとりうる値を $0,1,2,\dots$ と書くと,期待値は
    \[
        E(X)
        = \sum_{k=0}^\infty k \,\Pr(X = k)
    \]
    と書ける。ここで,各 $k \ge 1$ に対して $k \ge 1$ であることから
    \[
        E(X)
        = \sum_{k=0}^\infty k \,\Pr(X = k)
        \;\ge\; \sum_{k=1}^\infty 1 \cdot \Pr(X = k)
        = \sum_{k=1}^\infty \Pr(X = k).
    \]
    右辺の和は
    \[
        \sum_{k=1}^\infty \Pr(X = k)
        = 1 - \Pr(X = 0)
    \]
    なので,
    \[
        E(X) \;\ge\; 1 - \Pr(X = 0)
    \]
    が得られる。

    ここで仮定 $E(X) < 1$ を用いる。
    もし $\Pr(X = 0) = 0$ だとすると,
    \[
        1 - \Pr(X = 0) = 1
    \]
    だから,上の不等式から
    \[
        E(X) \;\ge\; 1
    \]
    が従う。これは仮定 $E(X) < 1$ と矛盾する。したがって
    \[
        \Pr(X = 0) > 0
    \]
    でなければならない。

    これは,「$\mathcal{E}$ のどの事象も起こらない」($X=0$)という場合の確率が
    正であることを意味する。したがって定理の主張が示された。
\end{proof}
