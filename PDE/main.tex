\documentclass[12pt]{jsarticle}
\usepackage[margin=15mm]{geometry}
\usepackage{amsmath,amssymb}

\begin{document}
\begin{center}
    {\LARGE \textbf{偏微分方程式 第8回課題}} \\[10pt]
    {\large 23TB8303 \quad 高橋知也}
\end{center}
\hrule
\vspace{2em}

\textbf{課題8-1}

1次元熱伝導方程式$\frac{\partial \theta(t,x)}{\partial t} = \kappa \frac{\partial^2 \theta(t,x)}{\partial x^2}$
の非自明な解を求めよ.ただし,境界条件は$\theta(t,0)=\theta_0$, $\theta(t,L)=\theta_L$とし,初期条件は以下を指定する.

\

A. 初期条件 $\theta(0,x)=\sin(\frac{3\pi}{L}x)$である時の解を求めよ.

B. 初期条件 $\theta(0,x)=x(L-x)$である時の解を求めよ.


\subsection*{1. 公式}

上の問題に対し,一般解は
\begin{equation}
    \theta(t,x)
    = \theta_s(x)
    + \sum_{n=1}^{\infty}
    C_n e^{-\kappa m_n t}
    \sin\!\bigl(\sqrt{m_n}\,x\bigr)
    \label{eq:general}
\end{equation}
で与えられる.ここで
\begin{align}
    \theta_s(x)
     & = \frac{\theta_L-\theta_0}{L}\,x + \theta_0,
    \label{eq:theta_s}                              \\[2mm]
    m_n
     & = \left(\frac{n\pi}{L}\right)^2,
    \label{eq:mn}                                   \\[2mm]
    C_n
     & = \frac{2}{L}
    \int_0^L
    \bigl(f(x)-\theta_s(x)\bigr)
    \sin\!\left(\frac{n\pi}{L}x\right)\,dx.
    \label{eq:Cn_def}
\end{align}
また
\begin{align}
    I_f
     & := \frac{2}{L} \int_0^L
    f(x)\,
    \sin\!\left(\frac{n\pi}{L}x\right)\,dx,
    \label{eq:If_def}          \\[2mm]
    I_{\theta_s}
     & := \frac{2}{L} \int_0^L
    \theta_s(x)\,
    \sin\!\left(\frac{n\pi}{L}x\right)\,dx
    \label{eq:Iths_def}
\end{align}
とおくと,\eqref{eq:Cn_def} より
\begin{equation}
    C_n = I_f - I_{\theta_s}
    \label{eq:Cn_If_Iths}
\end{equation}
である.

以下ではまず $I_{\theta_s}$ を一般の $n$ について求める.
その後,各初期条件について $I_f$ を計算し,\eqref{eq:Cn_If_Iths} から $C_n$ を得る.

\subsection*{2. $\boldsymbol{I_{\theta_s}}$ の計算}

\eqref{eq:theta_s} を用いると
\[
    I_{\theta_s}
    = \frac{2}{L}\int_0^L
    \left(
    \theta_0
    + \frac{\theta_L-\theta_0}{L}x
    \right)
    \sin\!\left(\frac{n\pi}{L}x\right)\,dx.
\]
ここで
\[
    k := \frac{n\pi}{L},\qquad
    \alpha := \frac{\theta_L-\theta_0}{L}
\]
とおくと
\[
    I_{\theta_s}
    = \frac{2}{L}
    \left(
    \theta_0 \int_0^L \sin(kx)\,dx
    + \alpha \int_0^L x\sin(kx)\,dx
    \right)
\]
となる.

\paragraph{(1) $\displaystyle \int_0^L \sin(kx)\,dx$ の計算}

\[
    \int \sin(kx)\,dx
    = -\frac{1}{k}\cos(kx)
\]
より,
\[
    \int_0^L \sin(kx)\,dx
    = \left[-\frac{1}{k}\cos(kx)\right]_0^L
    = -\frac{1}{k}\cos(kL)
    +\frac{1}{k}\cos(0)
    = \frac{1-\cos(kL)}{k}.
\]
$kL = n\pi$ なので
\[
    \cos(kL)=\cos(n\pi)=(-1)^n
\]
となり,
\[
    \int_0^L \sin(kx)\,dx
    = \frac{1-(-1)^n}{k}.
\]

\paragraph{(2) $\displaystyle \int_0^L x\sin(kx)\,dx$ の計算}

部分積分を用いる.
\[
    \int_0^L x\sin(kx)\,dx
    = \left[-\frac{x}{k}\cos(kx)\right]_0^L
    + \frac{1}{k}\int_0^L\cos(kx)\,dx
\]
まず
\[
    \left[-\frac{x}{k}\cos(kx)\right]_0^L
    = -\frac{L}{k}\cos(kL)
    -\left(-\frac{0}{k}\cos(0)\right)
    = -\frac{L}{k}\cos(kL),
\]
次に
\[
    \int_0^L\cos(kx)\,dx
    = \left[\frac{1}{k}\sin(kx)\right]_0^L
    = \frac{1}{k}\sin(kL)-\frac{1}{k}\sin(0)
    = \frac{1}{k}\sin(kL).
\]
したがって
\[
    \int_0^L x\sin(kx)\,dx
    = -\frac{L}{k}\cos(kL)
    + \frac{1}{k}\cdot\frac{1}{k}\sin(kL)
    = -\frac{L}{k}\cos(kL)
    + \frac{1}{k^2}\sin(kL).
\]
$kL=\dfrac{n\pi}{L}\cdot L = n\pi$ なので
\[
    \cos(kL)=\cos(n\pi)=(-1)^n,\qquad
    \sin(kL)=\sin(n\pi)=0
\]
より
\[
    \int_0^L x\sin(kx)\,dx
    = -\frac{L}{k}(-1)^n
    + \frac{1}{k^2}\cdot 0
    = -\frac{L}{k}(-1)^n.
\]

\paragraph{(3) $I_{\theta_s}$ のまとめ}

(1),(2) を $I_{\theta_s}$ に代入すると
\begin{align*}
    I_{\theta_s}
     & = \frac{2}{L}
    \left[
        \theta_0 \cdot \frac{1-(-1)^n}{k}
        + \alpha \cdot
        \left(
        -\frac{L}{k}(-1)^n
        \right)
    \right]           \\
     & = \frac{2}{Lk}
    \left[
        \theta_0(1-(-1)^n)
        - \alpha L(-1)^n
        \right].
\end{align*}
$\alpha L = \theta_L-\theta_0$ なので
\[
    -\alpha L(-1)^n
    = -(\theta_L-\theta_0)(-1)^n
    = -\theta_L(-1)^n + \theta_0(-1)^n.
\]
したがって
\begin{align*}
    I_{\theta_s}
     & = \frac{2}{Lk}
    \left[
        \theta_0(1-(-1)^n)
        -\theta_L(-1)^n
        +\theta_0(-1)^n
    \right]           \\
     & = \frac{2}{Lk}
    \left[
        \theta_0
        -\theta_L(-1)^n
        \right].
\end{align*}
ここで $Lk = n\pi$ だから
\begin{equation}
    I_{\theta_s}
    = \frac{2}{n\pi}
    \bigl(
    \theta_0 - \theta_L(-1)^n
    \bigr).
    \label{eq:Iths_final}
\end{equation}

\subsection*{A. 初期条件 $\boldsymbol{\theta(0,x)=\sin(\frac{3\pi}{L}x)}$ の場合}

$I_f$ を計算する.
\[
    I_f
    = \frac{2}{L}\int_0^L
    \sin\left(\frac{3\pi}{L}x\right)
    \sin\left(\frac{n\pi}{L}x\right)\,dx.
\]
ここで,$n=3$ のとき
\[
    I_f
    = \frac{2}{L}\int_0^L
    \sin^2\left(\frac{3\pi}{L}x\right)\,dx
    = \frac{2}{L}\int_0^L
    \frac{1-\cos\left(\frac{6\pi}{L}x\right)}{2}\,dx
\]
\[
    = \frac{1}{L}\int_0^L
    \left(1-\cos\left(\frac{6\pi}{L}x\right)\right)\,dx.
\]
ここで積分を分けて
\[
    I_f
    = \frac{1}{L}
    \left(
    \int_0^L 1\,dx
    - \int_0^L \cos\left(\frac{6\pi}{L}x\right)\,dx
    \right).
\]
まず
\[
    \int_0^L 1\,dx
    = \left[x\right]_0^L = L,
\]
次に
\[
    \int_0^L \cos\left(\frac{6\pi}{L}x\right)\,dx
    = \left[\frac{L}{6\pi}\sin\left(\frac{6\pi}{L}x\right)\right]_0^L
    = \frac{L}{6\pi}\bigl(\sin 6\pi - \sin 0\bigr)
    = 0.
\]
したがって
\[
    I_f
    = \frac{1}{L}(L - 0) = 1.
\]

$n\neq 3$ のとき
\[
    I_f
    = \frac{2}{L}\int_0^L
    \sin\left(\frac{3\pi}{L}x\right)
    \sin\left(\frac{n\pi}{L}x\right)\,dx.
\]
$x=\dfrac{Ly}{\pi}$ とおくと $dx=\dfrac{L}{\pi}dy$ なので
\[
    I_f
    = \frac{2}{\pi}\int_0^{\pi}
    \sin(3y)\sin(ny)\,dy.
\]
三角関数の積の公式
\[
    \sin a \sin b
    = \frac{1}{2}\bigl(\cos(a-b)-\cos(a+b)\bigr)
\]
を用いると
\begin{align*}
    I_f
     & = \frac{2}{\pi}\int_0^{\pi}
    \sin(3y)\sin(ny)\,dy           \\[1mm]
     & = \frac{2}{\pi}\int_0^{\pi}
    \frac{1}{2}\Bigl(
    \cos\bigl((3-n)y\bigr)
    - \cos\bigl((3+n)y\bigr)
    \Bigr)\,dy                     \\[1mm]
     & = \frac{1}{\pi}
    \int_0^{\pi}
    \Bigl(
    \cos\bigl((3-n)y\bigr)
    - \cos\bigl((3+n)y\bigr)
    \Bigr)\,dy.
\end{align*}
それぞれ積分して
\begin{align*}
    I_f
     & = \frac{1}{\pi}
    \left[
        \frac{1}{3-n}\sin\bigl((3-n)y\bigr)
        - \frac{1}{3+n}\sin\bigl((3+n)y\bigr)
        \right]_{y=0}^{y=\pi}.
\end{align*}
$3\pm n$ は整数だから
\[
    \sin\bigl((3\pm n)\pi\bigr)=0,\qquad
    \sin(0) = 0
\]
となる.したがって
\[
    I_f
    = \frac{1}{\pi}
    \left(
    \frac{1}{3-n}\cdot 0
    - \frac{1}{3+n}\cdot 0
    \right)
    = 0.
\]

よって
\begin{equation}
    I_f
    = \begin{cases}
        1 & (n=3)     \\[4pt]
        0 & (n\neq 3)
    \end{cases}
    \label{eq:If_initial_A}
\end{equation}

\paragraph{$C_n$ の計算}
\eqref{eq:Cn_If_Iths} と \eqref{eq:Iths_final}, \eqref{eq:If_initial_A} より
\begin{equation}
    C_n
    = I_f - I_{\theta_s}
    = I_f - \frac{2}{n\pi}
    \bigl(
    \theta_0 - \theta_L(-1)^n
    \bigr).
    = \begin{cases}
        1 - \dfrac{2}{3\pi}
        \bigl(
        \theta_0 + \theta_L
        \bigr) & (n=3)     \\[12pt]
        \dfrac{2}{n\pi}
        \bigl(
        \theta_L(-1)^n-\theta_0
        \bigr) & (n\neq 3)
    \end{cases}
\end{equation}

よって,初期条件 $\theta(0,x)=\sin(\frac{3\pi}{L}x)$ の場合の解は
\[
    \theta(t,x)
    = \theta_s(x)
    + \sum_{n=1}^{\infty}
    C_n e^{-\kappa m_n t}
    \sin\!\bigl(\sqrt{m_n}\,x\bigr)
\]
\[
    = \frac{\theta_L-\theta_0}{L}x + \theta_0
    + \left[
        1 - \frac{2}{3\pi}
        \bigl(
        \theta_0 + \theta_L
        \bigr)
        \right]
    e^{-\kappa (\frac{3\pi}{L})^2 t}
    \sin\left(\frac{3\pi}{L}x\right)
\]
\[    + \sum_{\substack{n=1 \\ n\neq 3}}^{\infty}
    \frac{2}{n\pi}
    \bigl(
    \theta_L(-1)^n-\theta_0
    \bigr)
    e^{-\kappa (\frac{n\pi}{L})^2 t}
    \sin\left(\frac{n\pi}{L}x\right).
\]


ここで、$n=1$ と $n=2$ の項を分けて書くと$n \neq 3$の条件を外せるので、最終的に以下のようになる。


\[
    \boxed{%
        \begin{aligned}
            \theta(t,x)
             & = \frac{\theta_L-\theta_0}{L}\,x + \theta_0 \\[1mm]
             & \quad
            - \frac{2}{\pi}(\theta_0 + \theta_L)\,
            e^{-\kappa \left(\frac{\pi}{L}\right)^2 t}
            \sin\left(\frac{\pi}{L}x\right)                \\[1mm]
             & \quad
            + \frac{1}{\pi}(\theta_L - \theta_0)\,
            e^{-\kappa \left(\frac{2\pi}{L}\right)^2 t}
            \sin\left(\frac{2\pi}{L}x\right)               \\[1mm]
             & \quad
            + \left[
                1 - \frac{2}{3\pi}(\theta_0 + \theta_L)
                \right]
            e^{-\kappa \left(\frac{3\pi}{L}\right)^2 t}
            \sin\left(\frac{3\pi}{L}x\right)               \\[1mm]
             & \quad
            + \sum_{n=1}^{\infty}
            \frac{2}{(n+3)\pi}\,
            \bigl(\theta_L(-1)^{n+3}-\theta_0\bigr)\,
            e^{-\kappa \left(\frac{(n+3)\pi}{L}\right)^2 t}
            \sin\left(\frac{(n+3)\pi}{L}x\right).
        \end{aligned}}
\]


\pagebreak
\subsection*{B. 初期条件 $\boldsymbol{\theta(0,x)=x(L-x)}$ の場合}
$I_f$ を計算する.
\[
    I_f
    = \frac{2}{L}\int_0^L
    x(L-x)
    \sin\left(\frac{n\pi}{L}x\right)\,dx.
\]
ここで部分積分を用いる.
\begin{align*}
    \int_0^L x(L-x)
    \sin\left(\frac{n\pi}{L}x\right)\,dx
     & =
    \left[
        -x(L-x)\,
        \frac{L}{n\pi}
        \cos\left(\frac{n\pi}{L}x\right)
    \right]_0^L                          \\[2mm]
     & \quad
    + \frac{L}{n\pi}
    \int_0^L (L-2x)
    \cos\left(\frac{n\pi}{L}x\right)\,dx \\[2mm]
     & =
    \frac{L}{n\pi}
    \int_0^L (L-2x)
    \cos\left(\frac{n\pi}{L}x\right)\,dx \\[2mm]
     & =
    \frac{L}{n\pi}
    \left(
    L\int_0^L
    \cos\left(\frac{n\pi}{L}x\right)\,dx
    \right)
    - \frac{2L}{n\pi}
    \int_0^L
    x\cos\left(\frac{n\pi}{L}x\right)\,dx.
\end{align*}


\begin{align*}
    \int_0^L
    \cos\left(\frac{n\pi}{L}x\right)\,dx
     & =
    \left[
        \frac{L}{n\pi}
        \sin\left(\frac{n\pi}{L}x\right)
    \right]_0^L                            \\[1mm]
     & =
    \frac{L}{n\pi}
    \sin\left(\frac{n\pi}{L}L\right)
    - \frac{L}{n\pi}
    \sin\left(\frac{n\pi}{L}\cdot 0\right) \\[1mm]
     & =
    \frac{L}{n\pi}\sin(n\pi)
    - \frac{L}{n\pi}\sin 0                 \\[1mm]
     & =
    \frac{L}{n\pi}\cdot 0
    - \frac{L}{n\pi}\cdot 0                \\[1mm]
     & = 0,
\end{align*}

\pagebreak

\begin{align*}
    \int_0^L
    x\cos\left(\frac{n\pi}{L}x\right)\,dx
     & =
    \left[
        x\cdot \frac{L}{n\pi}
        \sin\left(\frac{n\pi}{L}x\right)
        \right]_0^L
    - \frac{L}{n\pi}
    \int_0^L
    \sin\left(\frac{n\pi}{L}x\right)\,dx \\[1mm]
     & =
    \frac{L}{n\pi}
    \Bigl(
    L\sin\left(\tfrac{n\pi}{L}L\right)
    - 0\cdot
    \sin\left(\tfrac{n\pi}{L}\cdot 0\right)
    \Bigr)
    - \frac{L}{n\pi}
    \int_0^L
    \sin\left(\frac{n\pi}{L}x\right)\,dx \\[1mm]
     & =
    \frac{L^2}{n\pi}\sin(n\pi)
    - \frac{L}{n\pi}
    \int_0^L
    \sin\left(\frac{n\pi}{L}x\right)\,dx \\[1mm]
     & =
    0
    - \frac{L}{n\pi}
    \int_0^L
    \sin\left(\frac{n\pi}{L}x\right)\,dx \\[2mm]
     & =
    - \frac{L}{n\pi}
    \left[
        -\frac{L}{n\pi}
        \cos\left(\frac{n\pi}{L}x\right)
    \right]_0^L                          \\[1mm]
     & =
    \frac{L^2}{(n\pi)^2}
    \left(
    \cos\left(\tfrac{n\pi}{L}L\right)
    - \cos\left(\tfrac{n\pi}{L}\cdot 0\right)
    \right)                              \\[1mm]
     & =
    \frac{L^2}{(n\pi)^2}
    \bigl(\cos(n\pi)-\cos 0\bigr)        \\[1mm]
     & =
    \frac{L^2}{(n\pi)^2}
    \bigl((-1)^n-1\bigr).
\end{align*}


したがって
\begin{align*}
    \int_0^L x(L-x)
    \sin\left(\frac{n\pi}{L}x\right)\,dx
     & = \frac{L}{n\pi}
    \left(
    L\cdot 0
    -2 \cdot
    \frac{L^2}{(n\pi)^2}
    \bigl((-1)^n-1\bigr)
    \right)                    \\[1mm]
     & = \frac{2L^3}{(n\pi)^3}
    \bigl(1-(-1)^n\bigr).
\end{align*}
よって
\begin{equation}
    I_f
    = \frac{2}{L}
    \cdot
    \frac{2L^3}{(n\pi)^3}
    \bigl(1-(-1)^n\bigr)
    = \frac{4L^2}{(n\pi)^3}
    \bigl(1-(-1)^n\bigr)
    \label{eq:If_initial_B}
\end{equation}
\pagebreak
\paragraph{$C_n$ の計算}
\eqref{eq:Cn_If_Iths} と \eqref{eq:Iths_final} , \eqref{eq:If_initial_B} より
\begin{align*}
    C_n
     & = I_f - I_{\theta_s}    \\
     & = \frac{4L^2}{(n\pi)^3}
    \bigl(1-(-1)^n\bigr)
    - \frac{2}{n\pi}
    \bigl(
    \theta_0 - \theta_L(-1)^n
    \bigr).
\end{align*}
$n$の偶奇によって場合分けすると,
\begin{equation}
    C_n
    = \begin{cases}
        - \dfrac{2}{n\pi}
        \bigl(
        \theta_0 - \theta_L
        \bigr) & (n:\text{偶数}) \\[12pt]
        \dfrac{8L^2}{(n\pi)^3}
        - \dfrac{2}{n\pi}
        \bigl(
        \theta_0 + \theta_L
        \bigr) & (n:\text{奇数})
    \end{cases}
\end{equation}

よって,初期条件 $\theta(0,x)=x(L-x)$ の場合の解は
\[
    \theta(t,x)
    = \theta_s(x)
    + \sum_{n=1}^{\infty}
    C_n e^{-\kappa m_n t}
    \sin\!\bigl(\sqrt{m_n}\,x\bigr)
\]
\[
    = \frac{\theta_L-\theta_0}{L}x + \theta_0
    + \sum_{\substack{n=1 \\ n:\text{偶数}}}^{\infty}
    \left(
    - \frac{2}{n\pi}
    \bigl(
        \theta_0 - \theta_L
        \bigr)
    \right)
    e^{-\kappa (\frac{n\pi}{L})^2 t}
    \sin\left(\frac{n\pi}{L}x\right)
\]
\[    + \sum_{\substack{n=1 \\ n:\text{奇数}}}^{\infty}
    \left(
    \frac{8L^2}{(n\pi)^3}
    - \frac{2}{n\pi}
    \bigl(
        \theta_0 + \theta_L
        \bigr)
    \right)
    e^{-\kappa (\frac{n\pi}{L})^2 t}
    \sin\left(\frac{n\pi}{L}x\right).
\]

ここで、偶数を$2n$、奇数を$2n-1$($n=1,2,3,\ldots$)と置き換えると$n$が偶数(奇数)という条件を外せるので、最終的な解は以下のようになる。
\[
    \boxed{%
        \begin{aligned}
            \theta(t,x)
             & = \frac{\theta_L-\theta_0}{L}x + \theta_0 \\[1mm]
             & \quad
            + \sum_{n=1}^{\infty}
            \left(
            - \frac{1}{n\pi}
                (\theta_0 - \theta_L)
            \right)
            e^{-\kappa \left(\frac{2n\pi}{L}\right)^2 t}
            \sin\left(\frac{2n\pi}{L}x\right)            \\[1mm]
             & \quad
            + \sum_{n=1}^{\infty}
            \left(
            \frac{8L^2}{\bigl((2n-1)\pi\bigr)^3}
            - \frac{2}{(2n-1)\pi}
                (\theta_0 + \theta_L)
            \right)
            e^{-\kappa \left(\frac{(2n-1)\pi}{L}\right)^2 t}
            \sin\left(\frac{(2n-1)\pi}{L}x\right).
        \end{aligned}
    }
\]



\end{document}
