\section*{Homework 21}

\paragraph{Homework 21.}
Show that the converse of Theorem 5 does not hold.
(In other words, give an example of random variables $X, Y$ with finite second moments such that
$\operatorname{Cov}(X,Y) = 0$ but $X$ and $Y$ are not independent.)
\medskip

\begin{proof}[解答]
    離散型確率変数 $X$ を
    \[
        X =
        \begin{cases}
            -1           & \text{with probability } \frac{1}{4}, \\
            \phantom{-}0 & \text{with probability } \frac{1}{2}, \\
            \phantom{-}1 & \text{with probability } \frac{1}{4}
        \end{cases}
    \]
    と定める.このとき $X$ は $0$ を中心に対称であり,
    \[
        \mathbb{E}(X)
        = (-1)\cdot\frac14 + 0\cdot\frac12 + 1\cdot\frac14
        = 0
    \]
    である.ここで $Y = X^2$ とおくと,
    \[
        Y =
        \begin{cases}
            0 & \text{with probability } \frac{1}{2}, \\
            1 & \text{with probability } \frac{1}{2}
        \end{cases}
    \]
    となる.共分散を計算すると,
    \[
        \operatorname{Cov}(X,Y)
        = \mathbb{E}(XY) - \mathbb{E}(X)\mathbb{E}(Y)
        = \mathbb{E}(X^3) - 0\cdot\mathbb{E}(Y)
        = \mathbb{E}(X^3)
    \]
    である.$X^3$ の期待値は
    \[
        \mathbb{E}(X^3)
        = (-1)^3\cdot\frac14 + 0^3\cdot\frac12 + 1^3\cdot\frac14
        = -\frac14 + 0 + \frac14
        = 0
    \]
    なので
    \[
        \operatorname{Cov}(X,Y) = 0
    \]
    が成り立つ.

    一方で $Y=X^2$ であるから,
    \[
        \Pr(X=1,Y=0)=0
    \]
    であるが,
    \[
        \Pr(X=1)\Pr(Y=0)=\frac14\cdot\frac12=\frac18\neq 0
    \]
    となる.したがって $X$ と $Y$ は独立ではない.

    以上より,Theorem~5 の逆は成り立たない.
\end{proof}
