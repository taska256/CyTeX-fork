\section*{Homework 23}

\paragraph{Homework 23.}
Complete the proof of Theorem 12 by proving the lower tail bound
\begin{equation}\label{eq:target}
    \Pr\!\bigl(X \le (1 - \delta)\mathbb{E}(X)\bigr)
    \le
    \left(
    \frac{e^{-\delta}}{(1 - \delta)^{1-\delta}}
    \right)^{\mathbb{E}(X)}
    \quad (0 < \delta < 1).
\end{equation}

\begin{proof}[解答]
    Theorem 12 の証明と同様に,$X = \sum_{i=1}^n X_i$ を互いに独立な指示変数の和とする.
    各 $i$ について,ある $p_i \in [0,1]$ が存在して
    \[
        X_i =
        \begin{cases}
            1 & \text{with probability } p_i,  \\
            0 & \text{with probability } 1-p_i
        \end{cases}
    \]
    と書ける.このとき,各 $X_i$ のモーメント母関数 $M_{X_i}(t)$ について
    \[
        M_{X_i}(t)
        = \mathbb{E}(e^{t X_i})
        = p_i e^t + (1-p_i)
        = 1 + p_i(e^t - 1)
        \le e^{p_i (e^t - 1)}
    \]
    が成り立つ.ここでは任意の実数 $x$ に対して $1+x \le e^x$ という不等式を用いた.
    これは,$e^x$ に対してマクローリンの定理を適用すると
    \[
        e^x = 1 + x + \frac{x^2}{2} e^{\theta x} \qquad (0<\theta<1)
    \]
    が成り立つので,
    \[
        e^x - (1+x)
        = \frac{x^2}{2} e^{\theta x} \ge 0
    \]
    となる.したがって,$1+x \le e^x$ が従う.

    よって Lemma 10 より $X$ のモーメント母関数 $M_X(t)$ は
    \begin{equation}\label{eq:MX-upper}
        M_X(t)
        = \prod_{i=1}^n M_{X_i}(t)
        \le \prod_{i=1}^n e^{p_i(e^t - 1)}
        = e^{(e^t - 1)\sum_{i=1}^n p_i}
        = e^{(e^t - 1)\mathbb{E}(X)}
    \end{equation}
    を満たす($t \in \mathbb{R}$ の任意の値でよい).

    ここで $0 < \delta < 1$ を固定し,
    \[
        a := (1-\delta)\mathbb{E}(X)
    \]
    とおく.$t < 0$ のとき,$X \le a \iff e^{tX} \ge e^{ta}$ であり,
    Markov の不等式から
    \begin{equation}\label{eq:markov-left}
        \Pr(X \le a)
        = \Pr\bigl(e^{tX} \ge e^{ta}\bigr)
        \le \frac{\mathbb{E}(e^{tX})}{e^{ta}}
        = \frac{M_X(t)}{e^{ta}}
    \end{equation}
    を得る.\eqref{eq:MX-upper} と \eqref{eq:markov-left} をあわせると
    \begin{align}
        \Pr\!\bigl(X \le (1-\delta)\mathbb{E}(X)\bigr)
         & \le \frac{e^{(e^t - 1)\mathbb{E}(X)}}{e^{t(1-\delta)\mathbb{E}(X)}} \notag \\
         & = e^{(e^t - 1 - t(1-\delta))\mathbb{E}(X)}
        \label{eq:before-choose-t}
    \end{align}
    が得られる.

    $t<0$について,$0<\delta<1$ のとき $\ln(1-\delta) < 0$ なので,
    \[
        t := \ln(1-\delta)
    \]
    とすると $e^t = 1-\delta$ であり,
    \[
        e^t - 1 - t(1-\delta)
        = (1-\delta) - 1 - (1-\delta)\ln(1-\delta)
        = -\delta - (1-\delta)\ln(1-\delta)
    \]
    となる.これを \eqref{eq:before-choose-t} に代入すると
    \begin{align*}
        \Pr\!\bigl(X \le (1-\delta)\mathbb{E}(X)\bigr)
         & \le e^{\bigl(-\delta
        - (1-\delta)\ln(1-\delta)\bigr)\mathbb{E}(X)}                           \\
         & = \Bigl( e^{-\delta} (1-\delta)^{-(1-\delta)} \Bigr)^{\mathbb{E}(X)} \\
         & = \left(
        \frac{e^{-\delta}}{(1-\delta)^{1-\delta}}
        \right)^{\mathbb{E}(X)}.
    \end{align*}
    これは\eqref{eq:target} の右辺である.したがって,式 \eqref{eq:target} が示された.
\end{proof}
