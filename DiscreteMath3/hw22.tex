\section*{Homework 22}

\paragraph{Homework 22.}
Let $p \in (0, 1)$. Consider a sequence of $n$ independent coin flips in which the coin will land on heads
with probability $p$ and tails with probability $1 - p$. Define $X$ to be the random variable that represents
the number of heads in the sequence. Show that for any $\varepsilon > 0$,
\begin{equation}\label{eq:upper-limit}
    \lim_{n \to \infty} \Pr\!\bigl(X \ge (1 + \varepsilon)\mathbb{E}(X)\bigr) = 0
\end{equation}
and
\begin{equation}\label{eq:lower-limit}
    \lim_{n \to \infty} \Pr\!\bigl(X \le (1 - \varepsilon)\mathbb{E}(X)\bigr) = 0.
\end{equation}
\medskip

\begin{proof}[解答]
    各試行 $i = 1, \dots, n$ について,
    \[
        X_i =
        \begin{cases}
            1 & \text{($i$回目が表)}, \\
            0 & \text{($i$回目が裏)}
        \end{cases}
    \]
    とおくと,$X_i$ は独立な指示変数で $\Pr(X_i = 1) = p$ である.
    表の総数 $X$ は
    \[
        X = \sum_{i=1}^n X_i
    \]
    と書け,$X$ はパラメータ $(n,p)$ の二項分布に従うので
    \begin{equation}\label{eq:EX-np}
        \mathbb{E}(X) = np
    \end{equation}
    である.

    \medskip
    \noindent\textbf{(1)について.}
    Chernoff 境界(Theorem~12)より,任意の $\varepsilon > 0$ に対して
    \begin{equation}\label{eq:chernoff-upper}
        \Pr\!\bigl(X \ge (1 + \varepsilon)\mathbb{E}(X)\bigr)
        \le
        \left(
        \frac{e^{\varepsilon}}{(1 + \varepsilon)^{1+\varepsilon}}
        \right)^{\mathbb{E}(X)}
    \end{equation}
    が成り立つ.ここで
    \begin{equation}\label{eq:def-ceps}
        c_\varepsilon
        := \frac{e^{\varepsilon}}{(1 + \varepsilon)^{1+\varepsilon}}
    \end{equation}
    とおくと,\eqref{eq:chernoff-upper} と \eqref{eq:EX-np} より
    \begin{equation}\label{eq:upper-bound-ceps}
        \Pr\!\bigl(X \ge (1 + \varepsilon)\mathbb{E}(X)\bigr)
        \le c_\varepsilon^{\mathbb{E}(X)}
        = c_\varepsilon^{np}
    \end{equation}
    を得る.

    さらに $0 < c_\varepsilon < 1$ を示す.分子・分母はともに正なので $c_\varepsilon > 0$ は自明である.
    対数をとると
    \begin{equation}\label{eq:log-ceps}
        \log c_\varepsilon
        = \varepsilon - (1+\varepsilon)\log(1+\varepsilon)
    \end{equation}
    であり,これを $\varepsilon$ で微分すると
    \begin{equation}\label{eq:dlog-ceps}
        \frac{d}{d\varepsilon}\log c_\varepsilon
        = \frac{d}{d\varepsilon}\bigl(\varepsilon - (1+\varepsilon)\log(1+\varepsilon)\bigr)
        = 1 - \bigl(\log(1+\varepsilon) + 1\bigr)
        = -\log(1+\varepsilon)
    \end{equation}
    となる.一方で,合成関数の微分より
    \[
        \frac{d}{d\varepsilon}\log c_\varepsilon
        = \frac{c_\varepsilon'}{c_\varepsilon}
    \]
    だから,\eqref{eq:dlog-ceps} を用いて
    \begin{equation}\label{eq:ceps-derivative}
        c_\varepsilon'
        = -c_\varepsilon \log(1+\varepsilon)
    \end{equation}
    を得る.ここで $c_\varepsilon > 0$ かつ $\varepsilon > 0$ のとき $\log(1+\varepsilon) > 0$ であるから,
    \eqref{eq:ceps-derivative} より
    \[
        c_\varepsilon' < 0 \quad (\varepsilon > 0)
    \]
    すなわち $c_\varepsilon$ は $\varepsilon > 0$ で単調減少である.
    また
    \[
        c_0 = \frac{e^0}{(1+0)^{1+0}} = 1
    \]
    であるから,単調減少性と合わせて任意の $\varepsilon > 0$ について
    \begin{equation}\label{eq:ceps-between}
        0 < c_\varepsilon < 1
    \end{equation}
    が従う.

    \eqref{eq:upper-bound-ceps} と \eqref{eq:ceps-between} より
    \[
        0 \le
        \Pr\!\bigl(X \ge (1 + \varepsilon)\mathbb{E}(X)\bigr)
        \le c_\varepsilon^{np}
    \]
    であり,$0 < c_\varepsilon < 1$ かつ $p \in (0,1)$ は定数なので
    \begin{equation}\label{eq:limit-cepsnp}
        \lim_{n \to \infty} c_\varepsilon^{np} = 0
    \end{equation}
    が成り立つ.はさみうちの原理から
    \begin{equation*}
        \eqref{eq:limit-cepsnp} \text{ と }
        0 \le
        \Pr\!\bigl(X \ge (1 + \varepsilon)\mathbb{E}(X)\bigr)
        \le c_\varepsilon^{np}
        \text{ より }
        \eqref{eq:upper-limit} \text{ が従う.}
    \end{equation*}

    \medskip
    \noindent\textbf{(2)について.}
    まず $0 < \varepsilon < 1$ の場合を考える.Chernoff 境界より
    \begin{equation}\label{eq:chernoff-lower}
        \Pr\!\bigl(X \le (1 - \varepsilon)\mathbb{E}(X)\bigr)
        \le
        \left(
        \frac{e^{-\varepsilon}}{(1 - \varepsilon)^{1-\varepsilon}}
        \right)^{\mathbb{E}(X)}
    \end{equation}
    が成り立つ.ここで
    \begin{equation}\label{eq:def-deps}
        d_\varepsilon
        := \frac{e^{-\varepsilon}}{(1 - \varepsilon)^{1-\varepsilon}}
    \end{equation}
    とおくと,\eqref{eq:chernoff-lower} と \eqref{eq:EX-np} より
    \begin{equation}\label{eq:lower-bound-deps}
        \Pr\!\bigl(X \le (1 - \varepsilon)\mathbb{E}(X)\bigr)
        \le d_\varepsilon^{\mathbb{E}(X)}
        = d_\varepsilon^{np}
    \end{equation}
    を得る.$\log d_\varepsilon$ を計算すると
    \begin{equation}\label{eq:log-deps}
        \log d_\varepsilon
        = -\varepsilon - (1-\varepsilon)\log(1-\varepsilon)
    \end{equation}
    であり,これを $\varepsilon$ で微分すると
    \begin{equation}\label{eq:dlog-deps}
        \frac{d}{d\varepsilon}\log d_\varepsilon
        = \frac{d}{d\varepsilon}\bigl(-\varepsilon - (1-\varepsilon)\log(1-\varepsilon)\bigr)
        = -1 - \bigl(-\log(1-\varepsilon) - 1\bigr)
        = \log(1-\varepsilon)
    \end{equation}
    となる.また合成関数の微分より
    \[
        \frac{d}{d\varepsilon}\log d_\varepsilon
        = \frac{d_\varepsilon'}{d_\varepsilon}
    \]
    だから,\eqref{eq:dlog-deps} を用いて
    \begin{equation}\label{eq:deps-derivative}
        d_\varepsilon'
        = d_\varepsilon \log(1-\varepsilon)
    \end{equation}
    を得る.ここで $d_\varepsilon > 0$ かつ $0 < \varepsilon < 1$ なら $0 < 1-\varepsilon < 1$ より
    $\log(1-\varepsilon) < 0$ であるから,\eqref{eq:deps-derivative} より
    \[
        d_\varepsilon' < 0 \quad (0 < \varepsilon < 1)
    \]
    となる.したがって $d_\varepsilon$ は $(0,1)$ 上で単調減少であり,
    \[
        d_0 = \frac{e^0}{(1-0)^{1-0}} = 1
    \]
    であるから,任意の $0 < \varepsilon < 1$ について
    \begin{equation}\label{eq:deps-between}
        0 < d_\varepsilon < 1
    \end{equation}
    が従う.

    \eqref{eq:lower-bound-deps} と \eqref{eq:deps-between} より
    \[
        0 \le
        \Pr\!\bigl(X \le (1 - \varepsilon)\mathbb{E}(X)\bigr)
        \le d_\varepsilon^{np}
    \]
    であり,$0 < d_\varepsilon < 1$ かつ $p \in (0,1)$ は定数なので
    \begin{equation}\label{eq:limit-depsnp}
        \lim_{n \to \infty} d_\varepsilon^{np} = 0
    \end{equation}
    となる.はさみうちの原理より
    \begin{equation*}
        \eqref{eq:limit-depsnp} \text{ と }
        0 \le
        \Pr\!\bigl(X \le (1 - \varepsilon)\mathbb{E}(X)\bigr)
        \le d_\varepsilon^{np}
        \text{ から,}
        0<\varepsilon<1 \text{ に対して } \eqref{eq:lower-limit} \text{ が成り立つ.}
    \end{equation*}

    一方,$\varepsilon \ge 1$ の場合を考える.

    まず $\varepsilon > 1$ のとき
    \[
        (1-\varepsilon)\mathbb{E}(X) < 0
    \]
    であり,$X \ge 0$ だから任意の $n$ について
    \[
        \Pr\!\bigl(X \le (1 - \varepsilon)\mathbb{E}(X)\bigr) = 0
    \]
    となる.従ってこの場合は極限は自明に $0$ である.

    次に $\varepsilon = 1$ のとき
    \[
        (1-\varepsilon)\mathbb{E}(X) = 0
    \]
    であり,
    \begin{equation}\label{eq:eps1-prob}
        \Pr\!\bigl(X \le (1 - \varepsilon)\mathbb{E}(X)\bigr)
        = \Pr(X \le 0)
        = \Pr(X = 0)
        = (1-p)^n
    \end{equation}
    となる.$0 < 1-p < 1$ なので $(1-p)^n \to 0$($n\to\infty$)であり,
    \eqref{eq:eps1-prob} からこの場合も極限は $0$ である.
    よってすべての $\varepsilon > 0$ について \eqref{eq:lower-limit} が成り立つ.

    以上より,任意の $\varepsilon > 0$ について
    \eqref{eq:upper-limit} および \eqref{eq:lower-limit} が示された.
\end{proof}
